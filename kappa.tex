\chapter{Introduction}
%\section{Cyber physical systems}
Semiconductor technology and "computing" have penetrated into every aspect of our society. There are few things that are manufactured today that does not in some way contain some kind of micro processor or computing device. Many of these things also have some kind of connectivity or network interface. These interfaces have proved to be a "vulnerability", thus making these devices susceptible to attack by a malicious entity.

Security vulnerabilities in computer systems is nothing new, but when more and more physical devices are connected to and controlled by a computer the ramifications of an attack is much larger.
Several noted cyber attacks against cyber-physical systems have been done. STUXNET, Black Energy, Triton etc.
In this work we naturally does not claim to solve all of the security issues within this domain. But we put forward some ides that might help alleviate some issues.

The nature of these deceives pose new security challenges. The foremost is the limited computational power, network capacity and energy. 
Other devices such as industrial control systems are not \emph{per say} constrained, but the task of controlling the process with hard real-time deadlines consume so much of the available resources that any security solution developed for this purpose must be light-weight. 

When discussing protocols protocols can be divided by two lines. Protocols can be grouped by functionality, eg. Access Control, Secure Communication, Time Synchronization etc. Or by domain, e.g Industrial Control systems, Aviation or general Internet protocols. 
In this thesis we look at protocols intended for use in constrained devices. 

Protocols designed in "silos" e.g industrial control protocols are not the same as protocols for electricity grids. 
Only lately have protocols been designed for general use. 



\section{Dissertation Outline}
We begin with an introduction to the field, and a look at some of the sub-problems.


\chapter{Background}
\subsection{Cyber-Physical Systems}
The first device connected to the Internet was a vending machine at Carnegie Mellon University in 1982. Almost 40 years later the number of cyber-physical devices are TODO Billions. 

\subsection{Wireless sensor networks}
History
Definitions

Projection of numbers
Communication

\subsubsection{Constrained devices}
Constrained devices (\cite{rfc7228}
Limited in CPU-Power, ROM and RAM memory, communication bandwidth and latency. Energy limited, because of battery power or energy harvesting.
Devices sleep to preserve battery.

These constraints make traditional security solutions developed for desktop and server computing environments undesirable. Public key encryption has to be hardware-accelerated in order to be feasible, x509 certificates require to much bandwidth and memory to be stored in RAM in a device. 


\subsection{Communication protocols}
Bluetooth
Zigbee and Z-Wave
TLS and DTLS
Object security protocols
CoAP

\subsubsection{IETF IoT}
Protocols and ideas

\subsection{IoT Data analytics}
Tie in to Paper A
IoT devices send data to servers for analytics, this accumulation of data can enable new analytic results 

\subsection{IoT device life-cycle}
Production
Initial deployment
Operation
Decommissioning
Transfer of ownership

\subsection{Digital Twins}
"cyber objects"
"A digital twin is a real time digital replica of a physical device" 	Bacchiega (2017)[Non academic reference]
Began in aviation, aircraft engines was one of the first applications. Has spread to Wind Turbines, HVAC (Building Automation) and Utilities. 
Already exists in manufacturing for process optimization. The technology can be used to improve security. In heterogeneous systems it is difficult to establish a picture of the system. A digital twin of the system can provide such a picture. This twin can be used for finding vulnerabilities, both by scanning for known vulnerabilities, static threat modeling and also to create a replica of the system to be used in a "Cyber Range". 

\chapter{Contributions and Conclusions}
\section{Contributions}
The following sections describe each contribution in more detail.

\subsection{\paperItitle}
In this paper we investigate the problem of ownership transfer. The process of transferring ownership of devices has been studied for RFID-tags but not for IoT devices. The core problem with ownership transfer is \emph{forward secrecy} and \emph{backwards secrecy} this means that after the transfer of ownership the new owner shall be unable to learn anything that has happened on the device or any message sent. The old owner shall not learn anything the new owner does after the transfer.

\subsection{\paperIItitle}
Wireless sensor networks are being deployed in larger numbers and scales. The data that is sampled is often sent to a remote server for analytics. This server might be owned by a third party or running in a cloud environment. Placing full trust in the analytics server is in this case undesirable. In this work we propose a new scheme of identity-privacy for data items. The scheme only uses symmetric key operations and is as such suitable for very constrained sensors.

\subsection{\paperIIItitle}
OSCORE is a protocol recently standardized by the IETF. In this work we evaluate the first constrained implementation of OSCORE against the state of the art IETF security protocol for constrained devices, CoAP. 
\subsection{\paperIVtitle}
We propose a security architecture for Industrial Control systems based on the concept of digital twins.

\section{Conclusions}
Maybe the real security mechanisms was the friends we made along the way.
\label{sec:kappa-conclusions}
{ \raggedright
\printbibliography[segment=\therefsegment,heading=bibintoc]
}
