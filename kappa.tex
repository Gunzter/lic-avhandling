\chapter{Introduction}
\section{Cyberphysical systems}


\section{Internet of Things}
Internet of Things has for a few years been one of the most talked about topics in computer science and computer security. The idea to connect *things* to a network and especially the internet has appealed to a lot of different people in different industries. The term 

"The S in IoT stands for security."



When discussing protocols protocols can be divided by two lines. Protocols can be grouped by functionality, eg. Access Control, Secure Communication, Time Synchronization etc. Or by domain, e.g Industrial Control systems, Aviation or general Internet protocols. 
In this thesis we look at protocols intended for use in constrained devices. 

Protocols designed in "silos" e.g industrial control protocols are not the same as protocols for electricity grids. 
Only lately have protocols been designed for general use. 


\subsection{Digital Twins}
"cyberobjects"
"A digital twin is a real time digital replica of a physical device" 	Bacchiega (2017)[Non academic reference]
Began in aviation, aircraft engines was one of the first applications. Has spread to Wind Turbines, HVAC (Building Automation) and Utilities. 
Already exsist in manufacturing for process optimization. The technology can be used to improve security. In hetrogenous systems it is difficult to establish a picture of the system. A digital twin of the system can provide such a picture. This twin can be used for finding vulnerabuilites, both by scanning for known vulnerabilites, static threat modeling and also to create a replica of the system to be used in a "Cyber Range". 

\section{Dissertation Outline}
\chapter{Background}
Internet of Things 
\subsection{Wireless sensor networks}
History
Definitions

Projection of numbers
Communication

\subsubsection{Constrained devices}
Constrained devices (reference IETF)
Limited in CPU-Power, ROM and RAM memory, communication bandwidth and latency. Energy limited, because of battery power or energy harvesting.
Devices sleep to preserve battery.
\cite{rfc7228}

\subsection{Communication protocols}
Bluetooth
Zigbee and Z-Wave
TLS and DTLS
Object security protocols
CoAP

\subsubsection{IETF IoT}
Protocols and ideas

\subsection{IoT Data analytics}
Tie in to Paper A
IoT devices send data to servers for analytics, this accumulation of data can enable new analytic results 

\subsection{IoT device life-cycle}
Production
Initial deployment
Operation
Decommissioning
Transfer of ownership

\chapter{Contributions and Conclusions}
\section{Contributions}
The following sections describe each contribution in more detail.

\subsection{\paperItitle}

\section{Conclusions}
\label{sec:kappa-conclusions}
{ \raggedright
\printbibliography[segment=\therefsegment,heading=bibintoc]
}
