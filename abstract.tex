Industrial Control Systems (ICS) are increasingly connected. While connecting systems increases flexibility productivity in ICS, it also introduces risks and security vulnerabilities. Media have reported several cyberattacks against ICS, and security is a top priority in the next generation of ICS. High availability requirements and severe consequences of cyber-attacks make securing ICS a challenging problem.

In this thesis, I present my work on security for ICS. My work include work on protocols for secure communications in small connected devices, sometimes called Industrial Internet of Things (IIoT). The contribution consists of: an evaluation of the recently standardized protocol OSCORE, in terms of efficiency, to investigate its suitability for constrained devices.
We also propose a novel way of encrypting sensor data in transit to a remote server for analytics so that the sender's identity remains hidden. 

The long lifetimes of ICS require the management of devices over an extended time. We have utilized the new concept Digital Twin, for a security architecture where physical components are synchronized to a Digital Twin, to keep track of their security status. 
Long lifetimes of devices in ICS also introduces the problem of how to deal with the ownership change. We have designed a protocol that transfers the ownership of IoT devices from one entity to another.