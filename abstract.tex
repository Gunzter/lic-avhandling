More and more devices are connected, and Industrial Control Systems (ICS) are among them. Connecting systems to the internet enables communication and optimize processes. However, it also introduces risks and security vulnerabilities. Several cyberattacks against ICS have been recognized, and security is a top priority in the next generation of ICS. The demands of high availability and the severe consequences of cyber attacks makes securing ICS a challenging problem.

In this thesis we present our work on security for ICS. We have investigated protocols for secure communications in small connected devices, sometimes called Industrial Internet of Things (IIoT). We have evaluated the recently standardized protocol OSCORE, in terms of efficiency. 
We also propose a novel way of encrypting sensor data, in transit to a remote sever for analytics, so that the senders identity remain hidden.

The long lifetimes of ICS require security management of devices over an extended time. We have utilized the new concept Digital Twin, for a security architecture where physical components are synchronized to a digital twin, to keep track of their security status.

Long lifetimes of devices in ICS also introduces the problem of how to deal with ownership change. We have designed a protocol that transfer the ownership of IoT devices from one entity to another.

